\documentclass[10pt]{academydoc}
\pagestyle{plain}

% Set Document Details
\doctype{proc} % spec, proc, tb (Specification, Procedure, Technical Bulletin)
\docname{Recommended Procedures for the Creation and Use of Digital Camera System Input Device Transforms (IDTs)}
\altdocname{Creation and Use of Digital Camera System Input Device Transforms (IDTs)}
% Sets the document name used in header - usually an abbreviated document title
\docnumber{P-2013-001}
\committeename{Academy Color Encoding System (ACES) Project Committee}
\docdate{March 29, 2016}
\summary{
In the Academy Color Encoding System, an Input Device Transform (IDT) processes non-color-rendered RGB image values from a digital camera system's capture of a scene lit by an assumed illumination source (the scene adopted white). The results of this process are white-balanced ACES RGB relative exposure values.

Camera system vendors are recommended to provide two IDTs for each product, one optimized for CIE Illuminant D55 (daylight) and a second optimized for the ISO 7589 Studio Tungsten illuminant. Camera system vendors may optionally provide additional IDTs for common illumination sources such as Hydrargyrum Medium-arc Iodide (HMI) and KinoFlo\textsuperscript{\textregistered{}} lamps.

The main body of this document provides a procedure for the creation and use of an IDT from the measured spectral responsivities of a digital camera system. Appendices provide examples of the use of the procedure, each illustrating a different engineering tradeoff. An additional appendix provides an alternative procedure for IDT creation if the spectral data required for the recommended procedure are unobtainable.
}

% Document Starts Here
\begin{document}

\maketitle

../template/notices.tex \newpage
% This file contains the content for the Revision History and 
\prelimsectionformat	% Change formatting to that of "Notices" section
\chapter{Revision History}
%% Modify below this line %%

\begin{tabularx}{\linewidth}{|l|l|X|}
    \hline
    Version & Date       & Description \\ \hline
    1.0     & 12/19/2014 & Initial Version
    \\ \hline
    1.0.1   & 04/24/2015 & Formatting and typo fixes \\ \hline
            & 03/29/2016 & Remove version number - to use modification date as UID \\ \hline
    &   &   \\ \hline
    &   &   \\ \hline
    &   &   \\ \hline
\end{tabularx}

\vspace{0.25in} % <-- DO NOT REMOVE
\chapter{Related Academy Documents} % <-- DO NOT REMOVE
\begin{tabularx}{\linewidth}{|l|X|}
    \hline
    Document Name & Description \\ \hline
    TB-2014-009 & Academy Color Encoding System (ACES) Clip-level Metadata File Format Definition and Usage \\ \hline
    S-2013-001 & ACESproxy -- An Integer Log Encoding of ACES Image Data \\ \hline
    S-2014-003 & ACEScc -- A Logarithmic Encoding of ACES Data for use within Color Grading Systems \\ \hline
    S-2014-006 & A Common File Format for Look-Up Tables \\ \hline
    & \\ \hline
\end{tabularx} \newpage

\tableofcontents \newpage

\input{acknowledgements} \newpage
% This file contains the content for the Introduction
\unnumberedformat	    % Change formatting to that of "Introduction" section
\chapter{Introduction} 	% Do not modify section title
%% Modify below this line %%

The Academy Color Encoding Specification (ACES) defines a common color encoding method using half-precision floating point values corresponding to linear exposure values encoded relative to a fixed set of extended-gamut RGB primaries. Many digital-intermediate color grading systems have been engineered assuming image data with primaries similar to the grading display and a logarithmic relationship between relative scene exposures and image code values.

This document describes a 32-bit single precision floating-point logarithm encoding of ACES known as ACEScct.

ACEScct uses values above 1.0 and below 0.0 to encode the entire range of ACES values. ACEScct values should not be clamped except as part of color correction needed to produce a desired artistic intent.

There is no image file container format specified for use with ACEScct as the encoding is intended to be transient and internal to software or hardware systems, and is specifically not intended for interchange or archiving.

For ACES values greater than 0.0078125, the ACEScct encoding function is identical to the pure-log encoding function of ACEScc. Below this point, the addition of a "toe" results in a more distinct "milking" or "fogging" of shadows when a lift operation is applied when compared to the same operation applied in ACEScc. This difference is in grading behavior is provided in response to colorist requests for behavior more similar to that of traditional legacy log film scan encodings. 

% This file contains the content for the Scope
\cleardoublepage
\numberedformat	
\chapter{Scope} 	% Do not modify section title
%% Modify below this line %%

This document describes a 32-bit floating point encoding of ACES for use within color grading systems. 

Equivalent functions may be used for implementation purposes as long as correspondence of grading parameters to this form of log implementation is properly maintained. This document is intended as a guideline to aid developers who are integrating an ACES workflow into a color correction system.
% This section contains the content for the References
\numberedformat
\chapter{References}
The following standards, specifications, articles, presentations, and texts are referenced in this text:
%% Modify below this line %%

SMPTE ST 2065-1:2012, Academy Color Encoding Specification (ACES)

SMPTE RP 177:1993, Derivation of Basic Television Color Equations

% This section contains the content for the Terms and Definitions
\numberedformat
\chapter{Terms and Definitions}
The following terms and definitions are used in this document.
%% Modify below this line %%

\term{Academy Color Encoding Specification (ACES)}
RGB color encoding for exchange of image data that have not been color rendered, between and throughout production and postproduction, within the Academy Color Encoding System. ACES is specified in SMPTE ST 2065-1.

\term{American Society of Cinematographers Color Decision List (ASC CDL)}
A set of file formats for the exchange of basic primary color grading information between equipment and software from different manufacturers. ASC CDL provides for Slope, Offset and Power operations applied to each of the red, green and blue channels and for an overall Saturation operation affecting all three.
 \newpage
\input{procedure}

\begin{appendices}
	\appendixchapter{Application of ASC CDL parameters to ACEScct image data}{i}
\label{appendixA}

American Society of Cinematographers Color Decision List (ASC CDL) slope, offset, power, and saturation modifiers can be applied directly to ACEScct image data. To preserve the extended range of ACEScct values, no limiting function should be applied with ASC CDL parameters. The power function, however, should not be applied to any negative ACEScct values after slope and offset are applied. Slope, offset, and power are applied with the following function.

\note{ACEScct is not compatible with ASC CDL values generated on-set using the ACESproxy encoding. If there is a need to reproduce a look generated on-set where ACESproxy was used, ACEScc must be used in the dailies and/or DI environment to achieve a match.}

\begin{gather*} 
    ACEScct_{out} = \left\{ 
    \begin{array}{l r }
        ACEScct_{in} \times slope + offset; & \quad ACEScct_{slopeoffset} \leq 0 \\
        (ACEScct_{in} \times slope + offset)^{power}; & \quad ACEScct_{slopeoffset} > 0 \\
    \end{array} \right. \\ 
    \\
    \begin{array}{l}
    \text{Where:}\\
    ACEScct_{slopeoffset} = ACEScct_{in} \times slope + offset
    \end{array}
\end{gather*}

ASC CDL Saturation is also applied with no limiting function:

\begin{gather*}
    luma = 0.2126 \times ACEScct_{red} + 0.7152 \times ACEScct_{green} + 0.0722 \times ACEScct_{blue} \\
    \begin{aligned}
        ACEScct_{red} &= luma + saturation \times (ACEScct_{red} - luma) \\
        ACEScct_{green} &= luma + saturation \times (ACEScct_{green} - luma) \\        
        ACEScct_{blue} &= luma + saturation \times (ACEScct_{blue} - luma) \\ 
    \end{aligned}
\end{gather*}
    
	\input{appendixB}
\end{appendices}

\end{document}
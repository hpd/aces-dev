\documentclass[10pt]{academydoc}
\pagestyle{plain}

% Set Document Details
\doctype{tb} % spec, proc, tb (Specification, Procedure, Technical Bulletin)
\docname{Alternate ACES Viewing Pipeline User Experience}
\altdocname{Alternate ACES Viewing Pipeline User Experience}
% Sets the document name used in header - usually an abbreviated document title
\docnumber{TB-2014-013}
\committeename{Academy Color Encoding System (ACES) Project Committee}
\docdate{March 29, 2016}
\summary{
The majority of products that implemented pre-release ACES adopted an approach that combined the RRT and ODT into a single transform. It may useful to have one transform that has both ACES rendering components (from the RRT and ODT) that outputs the desired display colorimetry followed by a simple transform that converts that colorimetry into display code values. The ACES User Experience Working Group is developing an alternate UX proposal for products that wish to structure their viewing pipeline using this approach. This work is put forward for consideration as a possible recommended approach for a future ACES release.
}

% Document Starts Here
\begin{document}

\maketitle

../template/notices.tex \newpage
% This file contains the content for the Revision History and 
\prelimsectionformat	% Change formatting to that of "Notices" section
\chapter{Revision History}
%% Modify below this line %%

\begin{tabularx}{\linewidth}{|l|l|X|}
    \hline
    Version & Date       & Description \\ \hline
    1.0     & 12/19/2014 & Initial Version
    \\ \hline
    1.0.1   & 04/24/2015 & Formatting and typo fixes \\ \hline
            & 03/29/2016 & Remove version number - to use modification date as UID \\ \hline
    &   &   \\ \hline
    &   &   \\ \hline
    &   &   \\ \hline
\end{tabularx}

\vspace{0.25in} % <-- DO NOT REMOVE
\chapter{Related Academy Documents} % <-- DO NOT REMOVE
\begin{tabularx}{\linewidth}{|l|X|}
    \hline
    Document Name & Description \\ \hline
    TB-2014-009 & Academy Color Encoding System (ACES) Clip-level Metadata File Format Definition and Usage \\ \hline
    S-2013-001 & ACESproxy -- An Integer Log Encoding of ACES Image Data \\ \hline
    S-2014-003 & ACEScc -- A Logarithmic Encoding of ACES Data for use within Color Grading Systems \\ \hline
    S-2014-006 & A Common File Format for Look-Up Tables \\ \hline
    & \\ \hline
\end{tabularx} \newpage

\tableofcontents \newpage

% This file contains the content for the Introduction
\unnumberedformat	    % Change formatting to that of "Introduction" section
\chapter{Introduction} 	% Do not modify section title
%% Modify below this line %%

The Academy Color Encoding Specification (ACES) defines a common color encoding method using half-precision floating point values corresponding to linear exposure values encoded relative to a fixed set of extended-gamut RGB primaries. Many digital-intermediate color grading systems have been engineered assuming image data with primaries similar to the grading display and a logarithmic relationship between relative scene exposures and image code values.

This document describes a 32-bit single precision floating-point logarithm encoding of ACES known as ACEScct.

ACEScct uses values above 1.0 and below 0.0 to encode the entire range of ACES values. ACEScct values should not be clamped except as part of color correction needed to produce a desired artistic intent.

There is no image file container format specified for use with ACEScct as the encoding is intended to be transient and internal to software or hardware systems, and is specifically not intended for interchange or archiving.

For ACES values greater than 0.0078125, the ACEScct encoding function is identical to the pure-log encoding function of ACEScc. Below this point, the addition of a "toe" results in a more distinct "milking" or "fogging" of shadows when a lift operation is applied when compared to the same operation applied in ACEScc. This difference is in grading behavior is provided in response to colorist requests for behavior more similar to that of traditional legacy log film scan encodings.  \newpage

\input{sec-challenges}
\input{sec-structuring}
\input{sec-filtering}
\input{sec-roleoftherrt}
\input{sec-terminology}
\input{sec-mapping}

\end{document}
% This file contains the content for the Introduction
\unnumberedformat	    % Change formatting to that of "Introduction" section
\chapter{Introduction} 	% Do not modify section title
%% Modify below this line %%

The Academy Color Encoding Specification (ACES) defines a common color encoding method using half-precision floating point values corresponding to linear exposure values encoded relative to a fixed set of extended-gamut RGB primaries. Many digital-intermediate color grading systems have been engineered assuming image data with primaries similar to the grading display and a logarithmic relationship between relative scene exposures and image code values.

This document describes a 32-bit single precision floating-point logarithm encoding of ACES known as ACEScct.

ACEScct uses values above 1.0 and below 0.0 to encode the entire range of ACES values. ACEScct values should not be clamped except as part of color correction needed to produce a desired artistic intent.

There is no image file container format specified for use with ACEScct as the encoding is intended to be transient and internal to software or hardware systems, and is specifically not intended for interchange or archiving.

For ACES values greater than 0.0078125, the ACEScct encoding function is identical to the pure-log encoding function of ACEScc. Below this point, the addition of a "toe" results in a more distinct "milking" or "fogging" of shadows when a lift operation is applied when compared to the same operation applied in ACEScc. This difference is in grading behavior is provided in response to colorist requests for behavior more similar to that of traditional legacy log film scan encodings. 